% ---------------------------------------------------------
\documentclass[12pt]{article}

% ---------------------------
\usepackage{amsmath}
\usepackage{amsfonts}
\usepackage{amssymb}
\usepackage{amsthm}
\usepackage{amstext}
\usepackage{latexsym}
\usepackage{cases}
\usepackage{comment}
\usepackage{hyperref}
\usepackage{sectsty}
\usepackage{titlesec}
\usepackage[
        backend=biber,
        bibencoding=utf8,
        maxbibnames=99,
        maxcitenames=2,
        uniquelist=false,
        sorting=none,
        style=nature,
        giveninits=true,
]{biblatex}
\usepackage[
        body={6.5in, 9.0in},
        top=1.0in,
        left=1.0in,
        no head
]{geometry}
\usepackage{graphicx}
\usepackage{dirtree}

% ---------------------------
\urlstyle{same}
\hypersetup{
    colorlinks=true,    % false: boxed links; true: colored links
    linkcolor=blue,     % color of internal links
    citecolor=blue,     % color of links to bibliography
    filecolor=black,    % color of file links
    urlcolor=blue       % color of external links
}
\setlength{\parindent}{5mm}
\renewcommand{\thefootnote}{\fnsymbol{footnote}}
\footnotesep 1.2em
\sectionfont{\large}
\subsectionfont{\normalsize}
\titlespacing{\section}{0pt}{\parskip}{-0.5em}
\titlespacing{\subsection}{0pt}{\parskip}{-0.5em}
\titlespacing{\subsubsection}{0pt}{\parskip}{-0.5em}

% ---------------------------
\AtEveryBibitem{\clearfield{number}}
\addbibresource{bibfile.bib}

% ---------------------------------------------------------
\begin{document}

% ---------------------------
\parindent 0pt \parskip 1em
\centerline{\large{\bf{Cheyenne Noble Gas Workflow}}}
\vskip 1em
\centerline{\sl{P.~W.~Davidson and A.~M.~Seltzer, MIT and WHOI}}
\centerline{\sl{Fall 2023}}
% \parskip 0em
% \tableofcontents
% \parskip 1em

% ---------------------------
\begin{center}
	The general directory structure with relevant subdirectories listed for my \verb|Cheyenne| workflow is presented in the following trees. 
\end{center}

% ---------------------------
\section*{Home Directory}

The first is my \verb|home| directory, where the code and data for all experiments is stored:
\medskip
\dirtree{%
.1 /glade/u/home/pdavidson.
.2 matlab.
.3 toolboxes.
.4 gas-functions.  
.4 gas-toolbox-withNGisotopes.  
.4 tmm. 
.5 tmm-general.
.5 tmm-inertgasgasex-l13-xatm.
.2 petsc-master.
.2 tmm.
.3 data.
.4 default.
.4 inertgasgasex-l13-xatm.
.3 models.
.4 default.
.4 inertgasgasex-l13-xatm.
.5 lgmdefault.
.5 picdefault.
.3 tmm.
.4 models.
.5 current.
.6 inertgasgasex-l13-xatm.
.6 inertgasgasex-n16.
.6 inertgasgasex-s09.
.2 startup-*.
}
Everything in this directory is called by the experiment scripts in the \verb|work| directory, with the most important script really being the \verb|make_input_files_for_inert_gas_with_gasex_model.m|. If you can run that, then the model will be all set to run.
Of particular importance are: 
\begin{itemize}
	\item \verb|matlab/toolboxes/tmm-inertgasgasex-l13-xatm| toolbox which has all of the relevant scripts to run the Liang 2013 (L13) bubble model and processing the output data. I would just copy this into my experiment directory and add it to my \verb|MATLAB| path.
	\item \verb|tmm/data/tmm-inertgasgasex-l13-xatm| data which includes all the relevant \verb|OceanCarbon| data files for forcing the L13 model.
	\item \verb|tmm/models/tmm-inertgasgasex-l13-xatm| subdirectory which includes all of the TMM matrices and related data for both the PIC and the LGM. 
	\item \verb|tmm/tmm| which is the TMM GitHub repository in which I have made three new subdirectories in \verb|tmm/tmm/models/current| which include the three parameterizations that we care about, namely L13 as well as Nicholson 2016 and Stanley 2009.
\end{itemize}
Some general comments:
\begin{itemize}
	\item There should be a file in almost every (sub)-directory of the form \verb|ABOUT.txt| that explains everything, from where I got the data to when I got it to how I modified it and when. If not, just shoot me an \href{mailto:perrinwd@mit.edu}{email} and I think I can pull up some more information locally that explains what I was thinking at the time\ldots
	\item The \verb|startup_*| files can be run with \verb|source *| and those should load the appropriate modules that you need to run the model in \verb|Cheyenne| and process the data in \verb|Casper|.
\end{itemize}

You might note that with regard to your first point in your 24.11.2023 email, you cannot find the raw and simulated data from CMIP5 for the \verb|u10| velocity and the sea ice fraction \verb|sic| in this directory. 
This is because we were not able to pull from the WHOI CMIP5 server to \verb|Cheyenne|.
So, the data is stored at the following Google Drive link:
\href{https://drive.google.com/drive/folders/1gnJkjWTys_jUcoiROkvcyBvVMBaEskNw?usp=sharing}{link}
Be warned: this is a \emph{massive} directory. 
Additionally, you can find the processing scripts in the following GitHub repository:
\href{}{}.
Within the code should be how I organized the data into \verb|.nc| files. 
Let me know if I can help in understanding the code or how I organized the data.
Everything should be mostly self-contained and commented appropriately, but you never know\ldots

% ---------------------------
\section*{Work Directory}
The second is my \verb|work| directory where all experiments are performed:
\medskip
\dirtree{%
.1 /glade/work/pdavidson.
.2 cmip5.  
.2 noble-gas.
.2 noble-gas-new.
.2 tmm.
.3 experiments.
.4 all-models.
.5 circulation-change.
.5 lgm-wind-extent.
.5 lgm-wind-factor.
.5 lgm-xatm.
.5 pic-wind-freq.
.5 pmip3-sic-u10-prelim.
.4 inertgasgasex.
.4 inertgasgasex-l13-xatm.
.5 lgm-wind-perturbation.
.6 runs.
}


\end{document}
% ---------------------------------------------------------
